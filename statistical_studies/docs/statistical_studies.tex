% ****** Start of file apssamp.tex ******
%
%   This file is part of the APS files in the REVTeX 4.1 distribution.
%   Version 4.1r of REVTeX, August 2010
%
%   Copyright (c) 2009, 2010 The American Physical Society.
%
%   See the REVTeX 4 README file for restrictions and more information.
%
% TeX'ing this file requires that you have AMS-LaTeX 2.0 installed
% as well as the rest of the prerequisites for REVTeX 4.1
%
% See the REVTeX 4 README file
% It also requires running BibTeX. The commands are as follows:
%
%  1)  latex apssamp.tex
%  2)  bibtex apssamp
%  3)  latex apssamp.tex
%  4)  latex apssamp.tex
%
\documentclass[%
 reprint,
%superscriptaddress,
%groupedaddress,
%unsortedaddress,
%runinaddress,
%frontmatterverbose, 
%preprint,
%showpacs,preprintnumbers,
%nofootinbib,
%nobibnotes,
%bibnotes,
 amsmath,amssymb,
 aps,
%pra,
%prb,
%rmp,
%prstab,
%prstper,
%floatfix,
]{revtex4-1}

\usepackage{graphicx}% Include figure files
\usepackage{dcolumn}% Align table columns on decimal point
\usepackage{bm}% bold math
\usepackage{amsmath}% better dot placment
\usepackage{systeme}% systemes of equations
%\usepackage{hyperref}% add hypertext capabilities
%\usepackage[mathlines]{lineno}% Enable numbering of text and display math
%\linenumbers\relax % Commence numbering lines

%\usepackage[showframe,%Uncomment any one of the following lines to test 
%%scale=0.7, marginratio={1:1, 2:3}, ignoreall,% default settings
%%text={7in,10in},centering,
%%margin=1.5in,
%%total={6.5in,8.75in}, top=1.2in, left=0.9in, includefoot,
%%height=10in,a5paper,hmargin={3cm,0.8in},
%]{geometry}

% Personal definitions
\newcommand{\dvec}[1]{\dot{\vec{#1}}}
\newcommand{\grad}{\vec{\nabla}}
\newcommand{\intV}[1]{\int_{-\infty}^{\infty} #1 d^3x}
\newcommand{\intVdot}[1]{\int_{-\infty}^{\infty} #1 d^3\dot{x}}
\newcommand{\intVVdot}[1]{\int_{-\infty}^{\infty}\int_{-\infty}^{\infty} #1 d^3xd^3\dot{x}}


\begin{document}

\title{Statistical Studies: Experiment\# 11}% Force line breaks with \

\author{Luke A. Siemens}
\email{lsiemens@uvic.ca}

\date{\today}

\begin{abstract}
Two energy peaks of Cs-137 decay where observed with a multichannel analyzer and NaI detector, a Gamma-ray emission line with energy $679.1_{-2.7}^{+2.7}$KeV and a X-ray emission line with energy $14.88_{-0.15}^{0.15}$KeV was observed. The resolution of the multichannel analyzer was determined to be $6.943\pm0.039$.
\end{abstract}

\maketitle

\section{Introduction}

The firs component of this experiment focus on the statistical tool and techniques for analyzing data and statistical distribution. Specifically methods for fitting statistical distributions and techniques for determining if data is consistent with a theoretical model are investigated. In the second component of this experiment the statistical method investigated in the first section are applied to the practical problem of calibrating laboratory equipment, and error analysis of the subsequent measurements. In the course of this lab we will be investigating the Poisson distribution, the Gaussian distribution the $\chi^2$ distribution and the $\chi^2$ statistic and their application to the analysis of counting statistics and the calibration of a multichannel analyzer.

\section{Theory}
The statistic of counting random events is modeled by assuming the process in question can be described as a large number of simple binary events which each have a small probability of leading to a count being measured, where the number of binary event $n$ is $n\sim1/x$ each event having a probability of the binary event occurring is $p\sim1/x$ for $x<<1$. The probability that $\nu$ events will occur in a given interval is described by the binomial distribution,

\[
B_{n,p}(\nu)={n \choose \nu}p^\nu(1-p)^{n-\nu}
\]

In the limiting case of a continuous process $p -> 0$ and $n -> \infty$ the distribution simplifies to the Poisson distribution.

\begin{equation}
P_\mu(\nu)=\frac{e^{-\mu}\mu^\nu}{\nu!}
\label{poisson_dristribution}
\end{equation}

where the expected number of events $\bar{\nu}$ is $\bar{\nu}=\sum_{\nu=0}^\infty \nu P_\mu(\nu)=\mu$ and the variance is $\sigma_\nu^2=\sum_{\nu=0}^\infty(\nu-\mu)^2P_\mu(\nu) = \mu$. The Poisson distribution can be simplified in the limiting case where $\mu$ is large, in that case the Poisson distribution \eqref{poisson_dristribution} approaches a Gaussian distribution with mean $\mu$ and variance $\mu$. In this limit the distribution becomes,

\begin{equation}
P_\mu(\nu)\approx G_{\mu, \sqrt{\mu}}(\nu)=\frac{1}{\sqrt{2\pi\mu}}e^{-(\nu-\mu)^2/2\mu}
\label{poisson_gaussian_limit}
\end{equation}

\subsection{$\chi^2$ Test}
The $\chi^2$ test test is a method for determining the probability that a set of measurements is consistent with a given model assuming Gaussian errors. The $\chi^2$ test uses the $\chi^2$ statistic which in the case of discreet variables is defined as,

\[
\chi^2=\sum_{i}\left(\frac{O_i - E_i}{\sigma_i}\right)^2
\]

where $O_i$ is the observed value, $E_i$ is the expected value and $\sigma_i$ is the standard deviation. Given the expected value is distributed as a Poisson distribution with mean $\mu=E_i$ and variance $\sigma_i=\sqrt{E_i}$ then the $\chi^2$ statistic is,

\begin{equation}
\chi^2=\sum_{i}\frac{\left(O_i - E_i\right)^2}{E_i}
\label{chi_squared_statistic}
\end{equation}

Assuming the errors are Gaussian then the $\chi^2$ statistic is described by the associated $\chi^2$ distribution. The $\chi^2$ distribution which has the cumulative distribution function (CDF),

\begin{equation}
\text{Prob}_d(\chi^2 \leq \chi_o^2)=\frac{1}{2^{d/2}\Gamma(d/2)}\int_0^{\chi_o^2} t^{d/2-1}e^{-t/2}dt
\label{chi_squared_prob}
\end{equation}

So the probability of measuring a $\chi^2$ as large or larger than $\chi_o^2$ is $\text{Prob}_d(\chi^2 \geq \chi_o^2)=1-\text{Prob}_d(\chi^2 \leq \chi_o^2)$.

\subsection{Poisson errors and modified $\chi^2$ statistic}
The statistic given by equation \eqref{chi_squared_statistic} can be used with the $\chi^2$ assuming $O_i$ is described by a Gaussian distribution but that assumption is invalid when $E_i$ is not sufficiently large that the Poisson distribution can be approximated as a Gaussian distribution. If when $E_i$ is small the $\chi^2$ statistic must be modified to account for the asymmetry of the Poisson distribution. For the Gaussian distribution the probability of getting a value below the lower bound of a one sigma confidence interval is $p=\frac{1}{2} - \operatorname{erf}({\frac{1}{\sqrt{2}}})/2$, for the modified $\chi^2$ statistic instead of the standard deviation lets use the value $\left|\nu_o - \mu\right|$ where $\nu_o$ solves the equation $\text{Prob}_\mu\left(\nu \leq \nu_o\right)=p$ or $\text{Prob}_\mu\left(\nu \geq \nu_o\right)=1-p$ depending on whether the observed value is above or below the expected value. To evaluate these probabilities for $\mu$. The CDF of the poison distribution is,

\[
\text{Prob}_\mu\left(\nu \leq \nu_o\right)=\sum_{k=0}^{\nu_o}\frac{e^{-\mu}\mu^\nu}{\nu!}
\]

where $\nu_o$ is a positive integer. When $\mu \sim 1$ a continouse analouge to the Poisson CDF is needed to evaluate $\nu_o$ for the adjusted $\chi^2$ statistic. In the paper \textit{Continuous Counterparts of Poisson and Binomial Distributions and their Properites}\cite{continuous_poisson_paper} A. Ilienko defines a continuous analogue to the Poisson distribution, the CDF of this distribution is

\[
\text{Prob}_\mu\left(\nu \leq \nu_o\right)=\frac{\Gamma(\nu_o+1, \mu)}{\Gamma(\nu_o + 1)}
\]

where $\nu_o$ is a positive real number, $\Gamma(x, \lambda)$ is the incomplete Gamma function $\Gamma(x, \lambda)=\int_\lambda^\infty t^{x-1}e^{-t}dt$ and $\Gamma(x)=\Gamma(x, 0)$ is the Gamma function. Note that if $\nu_o$ is a positive integer then,

\[
\text{Prob}_\mu\left(\nu \leq \nu_o\right)=\frac{\Gamma(\nu_o+1, \mu)}{\Gamma(\nu_o + 1)}=\sum_{k=0}^{\nu_o}\frac{e^{-\mu}\mu^\nu}{\nu!}
\]

Using the continuous analogue to the Poisson distribution the adjusted $\chi^2$ statistic is defined as,

\[
\chi^2 = \sum_i \left(\frac{O_i - E_i}{\nu_{o_i} - E_i}\right)^2
\]

\[
\frac{\Gamma(\nu_{o_i}+1, E_i)}{\Gamma(\nu_{o_i} + 1)} = \begin{cases}
p_\text{mid} - p_\text{int}/2 & \text{if $O_i < E_i$}\\
p_\text{mid} + p_\text{int}/2 & \text{if $O_i > E_i$}
\end{cases}
\]

where $p_\text{mid} = \Gamma(E_i+1, E_i)/\Gamma(E_i + 1)$ and $p_\text{int}=\operatorname{erf}({\frac{1}{\sqrt{2}}})$. It should be noted that for $E_i > 30$ the relative difference between $\left|\nu_{o_i}-E_i\right|$ and $\sqrt{E_i}$ is less than ten percent and that if $p_\text{mid} < p_\text{int}/2$ the interval must be set to,

\[
\frac{\Gamma(\nu_{o_i}+1, E_i)}{\Gamma(\nu_{o_i} + 1)} = \begin{cases}
0 & \text{if $O_i < E_i$}\\
p_\text{int} & \text{if $O_i > E_i$}
\end{cases}
\]

\subsection{Energy resolution}
The resolution $R$ of a spectrometer is,

\begin{equation}
R = \frac{\delta E}{E}\cdot100
\label{energy_resolution}
\end{equation}

where $\delta E$ is the Full Width at Half Maximum (FWHM) of an energy peak and where $E$ is the energy at the maximum of the energy peak. The resolution $R$ gauges the detectors ability to resolve spectral features.

\section{Design and Results}
The experiment is split into two components, in the first section the distribution of counts produced by a sintilator(s) is measured and compared to theoretical distributions, in the second section the methods investigated for analyzing distributions is used to calibrate a multichannel analyzer and  to measure the resolution of the device.

\subsection{Section I}
The distribution produced by one sintilator for the cases of $\mu\approx5$ and $\mu\approx100$ where measured. We setup the sintilator with a $-1.5kV$ bias voltage. The signal from the sintilarot was passed into an amplifier and then through a discriminator. the signal from the discriminator was then passed to a timer/counter. Two sets of counts were collected with this setup each with 500 measurements. In the first set the discriminator cutoff was tuned to produce a low count rate $\mu\approx5$ (set \#1.1), in the second set the cutoff was tuned to produce a high count rate of $\mu\approx100$ (set \#1.2).

In the second part two sintilatorst where used to reject any signal that was not corrillated between the devices, so as to reduce the random noise from the sintelators. The sintelators where stacked vertically with approximately $6cm$ vertical separation. The signals from the two sintelators where then passed into two amplifiers and then into two discriminators. The two signals where then passed into a corrillator with corillation setting set to two. The output from the corrillator was then connected to a counter/timer. As in the first part two sets of data was taken with 500 measurements each, one with $\mu\approx5$ (set \#2.1) and the other with $\mu\approx100$ (set \#2.2).

\subsection{Section II}
The signal from a NaI detector with 750V power supply was passed to a \textit{Pocket MCA: 8000A}. The multichannel analyzer was set to record 1024 channels over two minutes for each of the samples. The multichannel analyzer was calibrated by fitting the energy peaks of Na-22 and Co-60 and using linear regression along with the known energy peaks of the Na-22 and Co-60 lines. Once calibrated a Gamma-ray and a X-ray peak of Cs-137 was measured and used to find the energy resolution of the multichannel analyzer.

\section{Analysis}
\subsection{section I}
The four sets of counts where fitted two three model distributions using $\chi^2$ minimization. The asymmetric Poisson errors where used in calculating the $\chi^2$ since number of counts per bin was small. The model distributions used where the Poisson distribution, the Gaussian distribution and the Gaussian limit of the Poisson distribution. When fitting no bins bellow the lowest nonzero or above the highest nonzero bin where used.

\begin{table}[!htbp]
\centering
\caption{Fit: Gaussian limit of the Poisson distribution}
\begin{tabular}{|l|l|l|l|l|l|}
\hline
set \# & $\mu$ in (counts) & dof & $\chi^2$ & p-value \\ \hline
1.2 & $143.55_{-0.48}^{+0.48}$ & 82 & $116.47$ & 0.7\% \\ \hline
1.1 & $4.567_{-0.086}^{+0.082}$ & 11 & $25.65$ & 0.7\% \\ \hline
2.1 & $4.924_{-0.082}^{+0.077}$ & 11 & $25.13$ & 0.9\% \\ \hline
2.2 & $96.09_{-0.38}^{+0.37}$ & 56 & $91.59$ & 0.2\% \\ \hline
\end{tabular}
\end{table}

\begin{table}[!htbp]
\centering
\caption{Fit: Poisson distribution}
\begin{tabular}{|l|l|l|l|l|l|}
\hline
set \# & $\mu$ in (counts) & dof & $\chi^2$ & p-value \\ \hline
1.2 & $143.42_{-0.52}^{+0.23}$ & 82 & $116.53$ & 0.7\% \\ \hline
1.1 & $4.602_{-0.099}^{+0.095}$ & 11 & $11.09$ & 43.6\% \\ \hline
2.1 & $4.884_{-0.098}^{+0.094}$ & 11 & $6.65$ & 82.7\% \\ \hline
2.2 & $95.95_{-0.36}^{+0.37}$ & 56 & $91.40$ & 0.2\% \\ \hline
\end{tabular}
\end{table}

\begin{table}[!htbp]
\centering
\caption{Fit: Gaussian distribution}
\begin{tabular}{|l|l|l|l|l|l|}
\hline
set \# & $\mu$ in (counts) & $\sigma$ in ($\text{counts}^2$) & dof & $\chi^2$ & p-value \\ \hline
1.2 & $143.31_{-0.60}^{+0.59}$ & $13.82_{-0.44}^{0.45}$ & 81 & $92.27$ & 18.41\% \\ \hline
1.1 & $4.545_{-0.124}^{+0.077}$ & $2.188_{-0.074}^{0.065}$ & 10 & $25.21$ & 0.5\% \\ \hline
2.1 & $4.834_{-0.123}^{+0.083}$ & $2.347_{-0.67}^{0.077}$ & 10 & $21.56$ & 0.2\% \\ \hline
2.2 & NA & NA & 55 & NA & NA \\ \hline
\end{tabular}
\end{table}

The fit of set \# 2.2 to the Gaussian distribution failed.

\subsection{section II}
When fitting the energy peaks, it was assumed that peak itself was Gaussian and that the local region around the peak could be described by a straight line. The model of the energy peak used in the fitting was,
\[
y = A e^{-(x - \mu)^2/c} + m (x - \mu) + b
\]

The best fit parameters where found by minimizing the $\chi^2$ using the function \textit{scipy.optimize.minimize} from the Python package \textit{scipy}. Since the minimum counts per channel exceeded three hundred counts the Gaussian limit of the Poisson distribution was used in calculating the $\chi^2$.

\begin{table}[!htbp]
\centering
\caption{Channel number and Energy of peaks}
\begin{tabular}{|l|l|l|l|l|l|}
\hline
sample & $\mu$ in (channel) & dof & $\chi^2$ & p-value & energy in (MeV) \\ \hline
Na-22 & $337.159_{-0.016}^{+0.016}$ & 55 & $200.92$ & 0.0\% & $0.511$ \\ \hline
Na-22 & $822.709_{-0.071}^{+0.071}$ & 113 & $121.90$ & 26.8\% & $1.274$ \\ \hline
Co-60 & $755.289_{-0.040}^{+0.041}$ & 82 & $97.97$ & 11.0\% & $1.17$ \\ \hline
Co-60 & $856.509_{-0.041}^{+0.041}$ & 100 & $104.25$ & 36.6\% & $1.33$ \\ \hline
Cs-137 & $443.914_{-0.016}^{+0.017}$ & 95 & $550.09$ & 0.0\% & $0.6612$ \\ \hline
Cs-137 & $22.130_{-0.012}^{+0.012}$ & 15 & $2024.62$ & 0.0\% & $0.0322$ \\ \hline
\end{tabular}
\end{table}

The Gamma-ray line energies are from \textit{Scintillation Spectrometry}\cite{scintillation}, the energy of the Ba $\text{K}\alpha$ line in the Cs-137 spectrum is from \textit{X-ray wavelengths}\cite{xray}.

Using the energy peaks from the Na-22 and Co-60 samples the multichannel analyzer was calibrated using linear regression. The calibration equation was found to be,

\[
E = \left(1.5748 \pm 0.0061 \text{KeV}/\text{channel}\right)x - 19.971 \pm 0.066 \text{KeV}
\]

Using the calibration equation the energies of the Cs-137 where found to be,

\begin{table}[!htbp]
\centering
\caption{Energy of Cs-137 lines}
\begin{tabular}{|l|l|l|l|l|l|}
\hline
$\mu$ in (channel) & observed energy & expected energy & percent error \\ \hline
$443.914_{-0.016}^{+0.017}$ & $679.1_{-2.7}^{+2.7}$KeV & $661.2$KeV & $+2.71$\% \\ \hline
$22.130_{-0.012}^{+0.012}$ & $14.88_{-0.15}^{0.15}$KeV & $32.2$KeV & $-53.8$\% \\ \hline
\end{tabular}
\end{table}

The using the $661.2$ KeV line of Cs-137 the parameter $c=323.33_{-0.51}^{+0.51}$ in $\text{channels}^2$ the resolution of the multichannel analyzer was determined to be $6.943\pm0.039$.

\section{Conclusion}
Using a multichannel analyzer to measure the gamma-ray emission of Cs-137 two lines where observed. A Gamma-ray emission line with energy $679.1_{-2.7}^{+2.7}$KeV was observed and found to have a $+2.71$\% deviation from the expected energy of $661.2$KeV. A X-ray emission line with energy $14.88_{-0.15}^{0.15}$KeV was observed and found to have a $-53.8$\% deviation from the expected energy of $32.2$KeV. The resolution of the multichannel analyzer was found to be $6.943\pm0.039$.

\pagebreak

\begin{thebibliography}{1}

\bibitem{continuous_poisson_paper} A. Ilienko, \textit{Continuous Counterparts of Poisson and Binomial Distributions and their Properties}, (Annales Univ. Sci. Budapest., Sect. Comp. 39(2013) 137-147).

\bibitem{scintillation} R. L. Heath, \textit{Scintillation Spectrometry: Gamma-ray Spectrum Catalogue}, (2nd Edition, Vol. 1, August 1964).

\bibitem{xray} J. A. Bearden, \textit{Xray wavelengths}, (Review of Modern Physics, January 1967, p 86-99).

\end{thebibliography}
 
\end{document}
%
% ****** End of file apssamp.tex ******
